\documentclass[12pt,a4paper]{article}

\usepackage[margin=1in]{geometry}
\usepackage[english]{babel}
\usepackage[T1]{fontenc}
\usepackage[utf8]{inputenc}

\usepackage{longtable}
\usepackage{tabularx}
\usepackage{booktabs}
\usepackage{array}
\usepackage{multirow}
\usepackage{graphicx}
\usepackage{xcolor}
\usepackage{enumitem}
\usepackage[parfill]{parskip}
\usepackage[title,titletoc]{appendix}
\usepackage{amsmath}
\usepackage{fnpos}
\usepackage{geometry}
\usepackage{csquotes}
\usepackage{tikz}
\usepackage{pdflscape}

\usetikzlibrary{positioning}
\usetikzlibrary{calc}
 \tikzstyle{main} = [rectangle, rounded corners, minimum height=0.7cm,text centered, draw=black, font=\scriptsize]
 \tikzstyle{process} = [rectangle, minimum height=0.7cm, text centered, draw=black, font=\scriptsize]
 \tikzstyle{note} = [rectangle, text centered, minimum height=1cm]
 \tikzstyle{arrow} = [->,>=stealth,auto,font=\scriptsize,node distance=3cm, main node/.style={circle,draw}]

\usepackage[hyphens]{url}

\usepackage[backend=biber,url=false,style=authoryear,sorting=nyt,bibstyle=numeric]{biblatex}
\addbibresource{chequeBounce.bib}

\usepackage[acronym,nonumberlist,nomain]{glossaries}
\makeglossary
\loadglsentries{acronyms.tex}

\AtBeginEnvironment{quote}{\smaller}
\makeFNbottom
\geometry{margin=1in}
\renewcommand{\baselinestretch}{1.25}
\setlength{\parskip}{1em}
\setlist{nosep}

\newcommand{\floattabu}[1]{
\vspace*{0.2in}
{\footnotesize
#1
}
\vspace*{0.2in}}

\usepackage[hidelinks,breaklinks]{hyperref}
\usepackage{cleveref}

\title{Characterising cheque dishonour cases in India: Causes for delays and policy implications}
\author{Devendra Damle\thanks{Devendra Damle is a researcher at the National Institute of Public Finance and Policy.}, Jitender Madaan\thanks{Jitender Madaan is an Associate Professor at the Indian Institute of Technology, Delhi.}, Karan Gulati\thanks{Karan Gulati is a researcher at the Vidhi Centre for Legal Policy.},\\ Manish Kumar Singh\thanks{Manish Kumar Singh is an Assistant Professor at the Indian Institute of Technology, Roorkee.} and Nikhil Borwankar\thanks{Nikhil Borwankar is a practicing advocate.}}

\begin{document}
\maketitle

\begin{abstract}
Cheque-dishonour cases constitute a significant portion of litigation in India. These cases are thought to be major contributors to overall pendency and delays in courts. The Supreme Court of India, in 2020, set up a committee and appointed Amici Curiae to formulate recommendations to reduce delays in cheque-dishonour cases. The recommendations target certain aspects of cheque-dishonour cases, however, empirical evidence supporting these recommendations is lacking. In this study we present an empirical analysis of the impact of six major characteristics of cases, which are the targets of the Supreme Court Committee and Amici Curiae's recommendations. We find that many of the recommendations --- more summary trials, clarifying jurisdictions, ensuring presence of the accused, reducing multiplicity of proceedings, and increasing the number of courts --- may indeed reduce the delays in cheque-dishonour cases. However, the recommendation to encourage more mediation is likely to cause more delays instead of reducing them. We thus demonstrate the importance of using empirical analysis to underpin important policy decisions in judicial procedure.
\end{abstract}

\newpage

Copyright Devendra Damle, Jitendar Madaan, Karan Gulati, Manish K Singh \& Nikhil Borwankar.

This working paper is being circulated for discussion and comment purposes. All rights reserved. Short sections of text, not to exceed two paragraphs, may be quoted without explicit permission from the copyright-holders provided that full credit, including copyright notice, is given to the source.

All content reflects the individual views of the authors.

\newpage
\tableofcontents

\newpage
\printglossaries

\newpage
\listoftables

\newpage
\section{Introduction}
\label{sec:introduction}
\input{introduction}

\section{Methodology}
\label{sec:methodology}
\input{methodology}

\section{Results}
\label{sec:results}
\input{results}

\section{Conclusion}
\label{sec:conclusion}
\input{conclusion}

\newpage

\section*{Acknowledgements}
The authors would like to thank the DAKSH Centre of Excellence for Law and Technology for their support. The authors are grateful to Sandhya PR and Surya Prakash BS for their extensive feedback on the study. The authors would like to thank Chockalingam Muthian for help with web-scraping. The authors would also like to thank Manaswini Rao and Pramod Rao for reviewing the study and their inputs. We also thank Abhishek Seth for his excellent research assistance. All errors are ours.

\section*{Funding}
This work was supported by the DAKSH Centre of Excellence (CoE) for Law and Technology at IIT Delhi.

\newpage
\printbibliography[heading=bibintoc]

\newpage
\begin{appendices}
\input{appendices}
\end{appendices}

\end{document}

%%% Local Variables:
%%% mode: latex
%%% TeX-master: t
%%% End:
