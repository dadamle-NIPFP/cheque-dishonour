India has a slow judiciary - courts are clogged with large backlogs.\footcite{moog1992delays, debroy2008justice, dutta2019modernise} As of 8th December 2021, over forty-one million cases were pending across district courts.\footcite{njdg2021} Slow judiciaries have adverse consequences on the structure and efficiency of markets and the quality of life of citizens.\footcite{world2004world, chemin2007impact} Therefore, minimising unnecessary judicial delays could help improve enforcement and enhance the overall rule of law. It is estimated that judicial delays cost India around 1.5\% of its GDP annually.\footcite{dey2016_cost} Against this backdrop, it is hardly surprising that tackling judicial delay has increasingly become a top priority for Indian judges and policymakers.

One reason for the large burden on courts is believed to be the alleged large share of \gls{ni} cases.\footnote{In particular, \S~138 of the Negotiable Instruments Act (Dishonour of cheque for insufficiency, etc., of funds in the account). Act 26 of 1881} As per the \gls{lci}, they represent 6.5\% and 7.8\% of all institutions and pendency in Indian courts, respectively.\footcite{lci2014_arrears} As per one order of the Supreme Court of India, they reflect more than 15\% of all criminal cases in the District Courts.\footcite{sc2020_makwanavstate} As per another order, they constitute 30\% of the total pendency in courts.\footcite[Similarly, a study published by the Department of Justice briefly touches on the burden of such cases on the judiciary and posits that they constitute 34\% of pending criminal cases in Maharashtra.][]{sc2020_138, mahadik2018_maharashtra}

Given the disagreement, any acceptance among policymakers that \gls{ni} cases (\textit{cheque dishonour cases} hereinafter) are the reason for India's slow judiciary is likely to lead to misguided solutions. The misrepresentation leads to diverting resources from other routes of judicial reform. Thus, one requires a more detailed analysis of the proportion of such cases, their causes, timelines, etc. This information can help judges and policymakers better target interventions.\footcite[For the importance of accurate judicial data, see][]{damle2020_ecourtsData, daksh2020_data, damle2020_land}

To this end, in 2020, the Supreme Court of India took on board a suo-motu case concerning the “expeditious trial of cases under section 138 of the Negotiable Instruments Act 1881”.\footcite{sc2020_138} Among other directions, the court (i) appointed a \gls{coe}\footnote{Headed by Hon’ble Mr Justice RC Chavan, former Judge of the Bombay High Court.} and (ii) appointed Amici Curiae to assist the court,\footnote{Mr Sidharth Luthra (Sr Advocate) and Mr K Parameshwar (Advocate).} to study processes to expedite disposal of complaints under \S~138 of the \gls{ni}. This is in a series of interventions targeted to understand cheque dishonour cases in India. As \cref{sec:history} shows, most knowledge and development concerning such cases in India has come from government institutions, i.e. the \gls{lci} and the judiciary.

Such knowledge is based on the assumption that cheque dishonour cases are a large burden on the judiciary and has thus focused on expediting disposal. For example, \gls{lci} (2008), by relying on newspaper reports concerning the proportion of cheque dishonour cases, recommended setting up Fast Track Magisterial Courts.\footcite{lci2008_138, bhan2015_placing} However, it did not define \textit{fast-track courts} or give guidance concerning how and where they would operate. Developing this work, Daksh (2017) examined the behaviour of cheque dishonour cases in the Indian courts by looking at over 67000 cases. In most cases, it found that resolution is delayed well beyond statutorily prescribed timelines and that certain banks and financial institutions are frequent complainants.\footcite{sridhar2017_cheque}

In 2020, the Amici appointed by the Supreme Court also made recommendations to expedite the disposal of cases. This included (i) increasing the use of pre- and post-summons mediation, (ii) expediting the service of summons to reduce absconsion, (iii) addressing the multiplicity of proceedings, etc.\footcite{amicus2020_submission} This study, based on orders of the Supreme Court and the report of the Amici Curiae, attempts to assess the effect of suggested interventions on the expeditious trial of cases under \S~138 of the \gls{ni}. It first answers important questions regarding the volume of such cases across differing courts. It then examines the determinants of case duration as identified by the Amici Curiae and the Supreme Court.

The rest of this paper is organised as follows: after this introduction, \cref{sec:history} elaborates the history of cheque dishonour provisions in India and their relation with judicial delays. \Cref{sec:methodology} describes the methodology, and \cref{sec:results} presents the results thereof. Finally, \cref{sec:conclusion} concludes the paper and presents the way forward.

\section{Cheque dishonour and judicial delays} \label{sec:history}

The \acrlong{ni} was enacted to define the law relating to promissory notes, bills of exchange, and cheques (for an explanation of \S~138, see Appendix \ref{app:understanding}).\footcite{ind1881_niAct} To increase the culture regarding the use of cheques, the Act was amended in 1988.\footcite{niAmend1988} A new chapter (\S\S~138 to 142) was incorporated for penalties in the event of dishonour of cheques due to insufficient funds in the account of the drawer of the cheque. \S~138 provided for the circumstances under a drawer can be penalised for the dishonour of a cheque.

However, by 2001, these provisions were thought not to have had the desired effect.\footcite{stdcomm2001_138niAct} While the punishment was thought to be inadequate, courts could not dispose of cases in a time-bound manner due to the large number of cases pending across the country. Given the large burden on courts, a Working Group was constituted the same year to review \S~138 of the \gls{ni} and make recommendations regarding the changes needed to effectively achieve the purpose of the section.\footcite{wg2001_138} In light of the recommendations of the Working Group, the government decided to bring further amendments to the Act. Among others, the amendments included:

\begin{enumerate}[label=(\alph*)]
 \item increasing the punishment from one year to two years;
 \item providing discretion to the court to waive the period of one month for taking cognisance of a case;
 \item prescribing the procedure for dispensing with preliminary evidence of the complainant;
 \item prescribing the procedure for service of summons via speed post or impanelled private couriers;
 \item providing for summary trials; and
 \item making the offence compoundable.
\end{enumerate}

The amendments were considered by the Lok Sabha Standing Committee on Finance, which recommended that given the large burden on courts, the proposed amendments be coupled with the creation of specialised courts for \S~138 cases.\footcite{stdcomm2001_138niAct} However, the subsequent Amending Act did not reference specialised courts.\footcite{niAmend2002} The Act was also amended in 2015 and 2018 to - clarify the appropriate area of jurisdiction where cheque dishonour cases can be filed,\footcite{niAmend2015} and grant power to courts to grant interim compensation, respectively.\footcite{niAmend2018}

\gls{lci} took up the proposal for specialised courts in 2008. According to the Commission, the credibility of the financial sector was facing setbacks due to the large pendency of dishonoured cheque cases. Relying on an array of judgments by the Supreme Court of India concerning speedy trials,\footcite{sc1978_khatoon, sc1981_champalal, sc2005_surinder, sc2008_krishna} the Commission recommended the introduction of fast track courts to address cases concerning dishonour of cheques under \S~138 of the Act. However, it did not comment on the number or expected workload of such courts.\footcite{lci2008_138} The need for additional courts was re-iterated by \gls{lci} in 2009.\footcite{lci2009_reforms}

While \gls{lci} has focused on the need for more courts, the Supreme Court has provided assistance in implementing the amendments. In 2014, noting that the prime reason for the delays in \S~138 case is the absence of the accused, the Court held that the magistrate should adopt a pragmatic and realistic approach such as issuing notices and summons via e-mail or push notification to ensure delivery.\footcite{sc2014_iba} In 2018, the court directed banks to give the details of the e-mail of the accused to the complainant for service through e-mail. The court also directed that: (i) cases must be dealt with summarily, (ii) the evidence of the complainant must be conducted within three months, (iii) an endeavour must be made to conclude the trial within six months, (iv) the trial, as far as practicable, must be held on a day-to-day basis, and (v) High Courts may pass additional localised directions for speedy disposal of cases.\footcite{sc2018_meters}

Lastly, in March 2020, the Supreme Court of India took on board a suo-motu case concerning the “expeditious trial of cases under section 138 of the Negotiable Instruments Act 1881”.\footcite{sc2020_138} As mentioned, the court-appointed Amici Curiae to assist in the matter, which presented its preliminary report to the court in October of the same year. The Amici made several suggestions to expedite trial, which among others, includes:

\begin{enumerate}[label=(\alph*)]
 \item Address jurisdictional issues;
 \item Address multiplicity of proceedings; and
 \item Expedite service of summons to reduce absconsion;
 \item Explore setting up specialised courts;
 \item Increase the use of pre- and post-summons mediation;
 \item Mandate presenting of a plausible defence before conversion from summary to summons trial; and
 \item Summon witnesses only when the accused presents a defence.\footcite{amicus2020_submission}
\end{enumerate}

Noticeably, the recommendations of the Amici Curiae were in line with previously accepted reasons for delays in disposal. While considering the recommendations, the Supreme Court recommended the Government and High Courts take necessary action where possible and held that they should be the subject matter of deliberation by the \gls{coe} appointed in the same case.\footnote{Headed by Hon’ble Mr Justice RC Chavan, former Judge of the Bombay High Court.}

%%% Local Variables:
%%% mode: latex
%%% TeX-master: "paper_chequeDishonour"
%%% End: