To examine the impact of case characteristics on case duration, we rely on two different measures: (1) Duration of the case (in days) and (2) Number of hearings required to dispose of. We perform a multivariate analysis: we regress the performance indicators (i.e., duration of the case and the number of hearings required to dispose of the case) on several potential explanatory variables. We also control for State and year fixed effects. For each performance measure, we estimate the following fixed effect regression model:

\begin{equation}\label{eq:fe1}
\begin{split}
Duration_i \ & or \ Number \ of \ hearings_i \\
& = \beta_1 \ D_1(Non-appearance_i) + \beta_2 \ D_2(Jurisdiction \ Issue_i) + \beta_3 \ D_3(Mediation_i) \\
& + \beta_4 \ D_4(Multiplicity_i) + \beta_5 \ D_5(Summons_i) + \beta_6 \ D_6(Contested_i) \\
&  + \alpha_s + Y_t + \epsilon_{ist}
\end{split}
\end{equation}

where $D_1$ is a dummy variable equal to 1 if the accused was not available / out of station/absconding or absent due to other reasons, $D_2$ is a dummy variable equal to 1 if the case refers to jurisdiction issues, $D_3$ is a dummy variable which takes the value 1 if the case was referred for mediation, $D_4$ is a dummy variable equal to 1 if the case suffers from a multiplicity of proceedings, $D_5$ is a dummy variable which takes the value 1 if the case is marked as summons trial, and $D_6$ is a dummy variable equal to 1 if the case was contested. We use the State and year dummies ($\alpha_s$ and $Y_t$ respectively) to control for the macro-economic reforms and environment.


%%% Local Variables:
%%% mode: latex
%%% TeX-master: "paper_chequeDishonour"
%%% End:
