\subsection{Data description}
\label{sec:data-description}
We first drew a random sample of 100,000 cases filed between 2014/01/01 and 2018/12/31 across India from the database published by Development Data Lab.\footcite{devdatalabs2021_eCourtsData} We then attempted to download the case-details for these 100,000 cases from the eCourts database. We were able to download case-details for approximately 89,000 cases. We then used regular expressions to check whether the ``Act Name'' field references the \gls{ni}. Then, we used regular expressions to search for references to the \gls{ni} in the texts of the interim orders and final orders, where available. We tagged the cases we found through this protocol as being related to \gls{ni}. Data availability and quality varies greatly across states. Consequently, we could not include all the states of India in our study. We identified the states to exclude from the study based on the following criteria:
\begin{itemize}
\item States/\glspl{ut} where we were unable to download sample data (this is because of data quality of the e-courts database);
\item States/\glspl{ut} that do not classify any cases as relating to the Negotiable Instruments Act;
\item States/\glspl{ut} where < 2\% of the final orders (as a proportion of total NI cases) are machine readable and in English.
\end{itemize}

Based on the aforementioned criteria, of the 29 states (including Jammu and Kashmir) and 7 \glspl{ut}, we dropped 21 states and 5 \glspl{ut}. Our final sample covers 8 states and 2 Union Territories, shown in Table \ref{tab:sample_desc}. For these 10 states and \glspl{ut}, we drew a random sample of 500,000 cases from the Development Data Labs database.\footcite{devdatalabs2021_eCourtsData} We then attempted to download case details for these 500,000 cases filed between 2014/01/01 and 2018/12/31. We were able to successfully download case-details for 4,68,855 cases. We removed cases from District and Sessions courts, since these courts hear appeals, and we are only interested in original matters. This gave us a sample of 3,63,762 cases. As with the preliminary analysis, we used pattern-matching based on regular expressions to identify cases related to \gls{ni}. In total, our sample has 48,191 \gls{ni} cases, out of these 10,693 are pending cases and 37,498 are disposed. Table \ref{tab:sample_desc} shows a summary of the sample. Our data shows that check-dishonour cases in India represent approximately 13.2\% of all cases in the courts (pending+disposed).

{\footnotesize \begin{longtable}{@{}lrrrr@{}}
\caption{Sample description}
\label{tab:sample_desc}\\
\toprule
\textbf{State} & \textbf{\gls{ni} cases} & \textbf{\%\gls{ni} cases} & \textbf{Non-\gls{ni} cases} & \textbf{Total cases}\\ \midrule
\endhead
Andhra Pradesh & 2640 & 12.4 & 18567 & 21207\\
Chandigarh & 731 & 34.9 & 1364 & 2095\\
Delhi & 5211 & 26.1 & 14742 & 19953\\
Goa & 399 & 12.8 & 2713 & 3112\\
Gujarat & 6756 & 12.0 & 49476 & 56232\\
Haryana & 5326 & 13.7 & 33542 & 38868\\
Himachal Pradesh & 1166 & 8.9 & 11948 & 13114\\
Karnataka & 11195 & 12.9 & 75807 & 87002\\
Maharashtra & 8880 & 9.7 & 82673 & 91553\\
Punjab & 5887 & 19.2 & 24739 & 30626\\
\textbf{Total} & \textbf{48191} & \textbf{13.2} & \textbf{315571} & \textbf{363762}\\ \bottomrule
\end{longtable}
}

\subsection{Approach to analysis}
\label{sec:approach-analysis}

Our objective is two-fold, to examine whether the interventions proposed by the Amicus Curiae are likely to have an effect on the duration of cheque-dishonour cases, and if so what is the magnitude of the effect. To that end, we first identified cases that had specific characteristics which are thought to increase the duration of a cheque-dishonour case. We then used a fixed-effects regression model to estimate the size of the effect of each of these characteristics on the total duration of the case. For the empirical analysis, we only used the 37,498 disposed cases.

\subsubsection{Text-mining}
\label{sec:text-mining}

We used pattern-matching based on regular expressions on the texts of
the interim and final orders for each case to identify cases with the
characteristics of interest. We supplemented this with the information
reported in the purpose of hearings and nature of disposal. The
characteristics and the strategy to identify whether it applies to a
given cases are as follows:

\begin{description}
\item [Non-appearance by the accused] --- We looked for references to Section 72
  CrPC, terms like accused not available / out of station / absconding
  / not present / absent, or that accused’s presence could not be
  secured, or references to the issuance of a bailable / non-bailable
  warrant to secure the presence of the accused. In addition to this
  we looked at references to non-appearance, absconsion, bailable /
  non-bailable warrants in the data on the purpose of hearings.
\item [Jurisdiction issue] --- We looked for references to lack of
  jurisdiction / lack of proper jurisdiction / beyond jurisdiction,
  references to section 142 of the NI Act, and citations of specific
  cases like Vijay Dhanuka, K S Joseph vs Phillips
  Carbon. Additionally, we also marked cases that had been
  transferred, since a vast majority of these cases get transferred
  because of jurisdictional issues, and cases marked as transferred in
  the hearings data and disposal data.
\item [Case referred to mediation] --- We looked for references to the case being
  referred to mediation / lok adalat / ADR, or references to the
  matter being settled mutually, or compromised. We also looked at
  references to the cases being sent to mediation in the hearings
  data, and references to the case concluding in the Lok Adalat or
  being settled mutually, or the parties reaching a compromise.
\item [Multiplicity of proceedings] --- We looked for references to multiple cheques
  being dishonored and multiple transactions, and references to
  Section 219 and/or Section 220 of the CrPC.
\item [Case being converted to a summons trial] --- We looked for references to the case being
  tried under section 143 of the NI Act, or Sections 262, 263, 264 or
  265 of the CrPc were marked, or these sections being applicable, or
  references to the case being tried summarily were marked as not
  summons trials. Cases where these were deemed not to apply, or that
  the case was being conducted normally, or it is undesirable to try
  the case summarily, or the aforementioned sections not being
  applicable were marked as summons trials. Further, cases where any
  hearing mentioned summons trial as the purpose, were marked as
  summons cases.
\item [Outcome of the case] --- We primarily used the data from the nature of
  disposal to identify the outcome. Additionally, we searched for
  references to the accused being found guilty, not guilty, convicted,
  acquitted or sentenced to fill gaps.
\end{description}

All of these case characteristics (except outcomes) are represented as
binary variables. As an example, if in a given case, if the parties
did not appear for a hearing, the Non-appearance variable takes the
value 1. If the parties appeared at every hearing, we mark it as
0. Similarly, for the Summons Trial variable, if the case was
converted to a summons trial it is marked as 1. Otherwise, it is
marked as 0. Table \ref{tab:case_chars} shows the state-wise number of
cases in which we were able to identify characteristics of interest.

{\footnotesize \centering
  \begin{longtable}{@{}lrrrrrr@{}}
  \caption{\mbox{Number of cases in which we were able to identify
    characteristics of interest}}
  \label{tab:case_chars}\\
  \toprule
  \textbf{State} & \textbf{Non} & \textbf{Summons} & \textbf{Mediation} & \textbf{Jurisdiction} & \textbf{Multiplicity} & \textbf{Total} \\
   & \textbf{Appearance} & \textbf{Trial} & & \textbf{Issue} & & \\
  \midrule
  \endhead
  Andhra Pradesh & 1674 & 814 & 260 & 210 & 124 & 2640 \\
  Chandigarh & 701 & 408 & 204 & 278 & 53 & 731 \\
  Delhi & 2062 & 1359 & 1026 & 1045 & 208 & 5211 \\
  Goa & 381 & 78 & 90 & 33 & 18 & 399 \\
  Gujarat & 5125 & 391 & 402 & 3059 & 107 & 6756 \\
  Haryana & 5214 & 3376 & 1731 & 2109 & 540 & 5326 \\
  Himachal Pradesh & 864 & 351 & 506 & 299 & 33 & 1166 \\
  Karnataka & 5850 & 2002 & 1846 & 953 & 410 & 11195 \\
  Maharashtra & 5933 & 963 & 2546 & 3831 & 135 & 8880 \\
  Punjab & 5821 & 3134 & 2100 & 2281 & 382 & 5887 \\
  \midrule
  \textbf{Total} & \textbf{33625} & \textbf{12876} & \textbf{10711} &
  \textbf{14098} & \textbf{2010} & \textbf{48191}\\\bottomrule
\end{longtable}
}

It must be noted that this is likely an underestimate because orders cannot be parsed in approximately 36\% of the cases. Further, some courts do not properly record details of the purpose of hearings and the nature of disposal. Cases in which information regarding these characteristics could not be found were marked as not having those particular characteristics. In our manual check of approximately 500 cases, we did not find any false positives or false negatives, however, the aforementioned data limitation also limits our ability to reliably check for false negatives. This in turn means there could be many more cases which have the aforementioned characteristics but could not be identified as having them.

\subsubsection{Regression model specifications}
\label{sec:model-selection}
To examine the impact of case characteristics on case duration, we rely on two different measures: (1) Duration of the case (in days) and (2) Number of hearings required to dispose of. We perform a multivariate analysis: we regress the performance indicators (i.e., duration of the case and the number of hearings required to dispose of the case) on several potential explanatory variables. We also control for State and year fixed effects. For each performance measure, we estimate the following fixed effect regression model:

\begin{equation}\label{eq:fe1}
\begin{split}
Duration_i \ & or \ Number \ of \ hearings_i \\
& = \beta_1 \ D_1(Non-appearance_i) + \beta_2 \ D_2(Jurisdiction \ Issue_i) + \beta_3 \ D_3(Mediation_i) \\
& + \beta_4 \ D_4(Multiplicity_i) + \beta_5 \ D_5(Summons_i) + \beta_6 \ D_6(Contested_i) \\
&  + \alpha_s + Y_t + \epsilon_{ist}
\end{split}
\end{equation}

where $D_1$ is a dummy variable equal to 1 if the accused was not available / out of station/absconding or absent due to other reasons, $D_2$ is a dummy variable equal to 1 if the case refers to jurisdiction issues, $D_3$ is a dummy variable which takes the value 1 if the case was referred for mediation, $D_4$ is a dummy variable equal to 1 if the case suffers from a multiplicity of proceedings, $D_5$ is a dummy variable which takes the value 1 if the case is marked as summons trial, and $D_6$ is a dummy variable equal to 1 if the case was contested. We use the State and year dummies ($\alpha_s$ and $Y_t$ respectively) to control for the macro-economic reforms and environment.


%%% Local Variables:
%%% mode: latex
%%% TeX-master: "paper_chequeDishonour"
%%% End:


%%% Local Variables:
%%% mode: latex
%%% TeX-master: "paper_chequeDishonour"
%%% End:
