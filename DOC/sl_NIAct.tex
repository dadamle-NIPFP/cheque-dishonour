\documentclass[10pt,aspectratio=169]{beamer}
\mode<presentation>
{
 \usetheme[progressbar=frametitle]{metropolis}
 \setbeamertemplate{navigation symbols}{}
 \setbeamertemplate{caption}[numbered]
 \setbeamertemplate{itemize subitem}{\tiny\raise1.5pt\hbox{\donotcoloroutermaths$\blacktriangleright$}}
}

\usepackage{booktabs}
\usepackage[T1]{fontenc}
\usepackage{graphicx}
\usepackage[latin1]{inputenc}
\usepackage{multicol}
\usepackage{multirow}
\usepackage{tabulary}
\usepackage{threeparttable}
\usepackage{longtable}

\title{Characterising cheque dishonour cases in India: Causes for delays and policy implications}
\author{Devendra Damle, Jitender Madaan, Karan Gulati, Manish Kumar Singh, and Nikhil Borwankar}
\date{\today}

\begin{document}
\frame{\maketitle}

\begin{frame}{The problem}
\begin{itemize}
 \item India has a slow judiciary - courts are clogged with large backlogs.
 \item Part of the reason is the (supposed) large share of Negotiable Instruments Act, 1881 (NI Act) cases.
 \item However, there is disagreement on the burden of these cases on courts. 
 \item To understand and curtail the pendency of NI Act Cases, the Supreme Court of India has taken on board a suo-motu case and appointed Amici Curiae.
 \item The Amici has made several recommendations, including:
 \begin{itemize}
 \item clarifying jurisdictional issues;
 \item encouraging pre-trial and post-summons mediation;
 \item limiting how many cases are converted into summons trials, 
 \item etc. 
 \end{itemize}
\end{itemize}
\end{frame}

\begin{frame}{The Study}
\begin{itemize}
 \item Our study attempts to assess the potential effect of the proposed interventions.
 \item The case characteristics examined were based on: \begin{enumerate}
 \item the importance of the characteristic, and
 \item the feasibility of finding reliable information about it in the eCourts data.
 \item[]
 \end{enumerate} \pause
 \item The six chosen case characteristics map onto the interventions proposed by the Supreme Court and Amici Curiae, and are:

 \begin{enumerate}
 \item The accused fails to appear before the court for at least one hearing;
 \item The case is converted to a summons trial;
 \item The case is referred to mediation;
 \item The case has jurisdictional issues and is transferred to another court, as a result;
 \item The case contains a multiplicity of proceedings --- either the dishonoured cheque was issued to satisfy multiple transactions, or multiple cheques were dishonoured;
 \item The case was contested.
\end{enumerate}
\end{itemize}
\end{frame}

\begin{frame}{Results i}
 \begin{enumerate}
 \item[]
 \item The burden on courts
 \begin{itemize}
 \item We short-listed 8 states and 2 union territories, and drew a random sample of 363,720 cases.
 \item NI Act cases represent approximately 13.2\% of courts' workload (pending and disposed).
 \end{itemize}
 \end{enumerate}

{\scriptsize \begin{longtable}{@{}lrrrr|c@{}}
\toprule
\textbf{State} & \multicolumn{1}{p{1.5cm}}{\textbf{NI Act cases}} & \multicolumn{1}{p{1.5cm}}{\textbf{Non-NI Act cases}} & \multicolumn{1}{p{1.5cm}}{\textbf{\%NI Act cases}} & \textbf{Total cases} & \multicolumn{1}{p{3cm}}{\textbf{Median days to dispose NI Act cases}}\\ \midrule
\endhead
Andhra Pradesh & 2638 & 18567 & 12.4 & 21205 & 451\\
Chandigarh & 731 & 1364 & 34.9 & 2095 & 453\\
Delhi & 5202 & 14742 & 26.1 & 19944 & 518\\
Goa & 399 & 2713 & 12.8 & 3112 & 352\\
Gujarat & 6752 & 49476 & 12.0 & 56228 & 323\\
Haryana & 5319 & 33542 & 13.7 & 38861 & 488\\
Himachal Pradesh & 1164 & 11948 & 8.9 & 13112 & 491\\
Karnataka & 11184 & 75807 & 12.9 & 86991 & 218\\
Maharashtra & 8875 & 82673 & 9.7 & 91548 & 520\\
Punjab & 5885 & 24739 & 19.2 & 30624 & 440\\
\midrule
\textbf{Total} & \textbf{48149} & \textbf{315571} & \textbf{13.2} & \textbf{363720} & \textbf{395}\\ \bottomrule
\end{longtable}
}
\end{frame}

\begin{frame}{Results ii}

\begin{enumerate}
 \setcounter{enumi}{1}
 \item Determinants of duration and number of hearings
 \begin{itemize}
 \item Controlling for the State-level effects, most intended targets of the Amici Curiae and Supreme Court do increase the duration of NI Act cases.
 \end{itemize}
\end{enumerate}

{\scriptsize \begin{longtable}{@{}p{2.5cm}rrrrr}
 \toprule
 \textbf{Characteristic} & \multicolumn{1}{p{1.7cm}}{\textbf{Number of cases}} &
 \multicolumn{1}{p{1.7cm}}{\textbf{As \% of NI Act cases}}
 & \multicolumn{1}{p{1.7cm}}{\textbf{As \% of total cases}}
 & \multicolumn{1}{p{1.7cm}}{\textbf{Effect on days to dispose}} &
 \multicolumn{1}{p{1.7cm}}{\textbf{Effect on hearings to dispose}}
 \\
 \midrule
 Non appearance of the accused & 33625 & 69.8\% & 9.2\% & +213 & +7.0 \\ \midrule
 Conversion to summons trial & 12876 & 26.7\% & 3.5\% & +111 & +7.2 \\ \midrule
 Mediation & 10711 & 22.2\% & 2.9\% & +108 & +3.3 \\ \midrule
 Jurisdictional issues & 14098 & 29.2\% & 3.9\% & +287 & +5.7 \\ \midrule
 Multiplicity of proceedings & 2010 & 4.2\% & 0.6\% & +171 & +10.0 \\ \midrule
 Contested & 8283 & 17.2\% & 2.3\% & --46 & +2.9 \\ \midrule
 \textbf{Total} & 48191 & 100.0\% & 13.2\% & N/A & N/A \\
 \bottomrule
 \\
\end{longtable}
}

\end{frame}

\begin{frame}{Way Forward}
 \begin{itemize}
 \item This study gives estimates of the scale of the impact of proposed interventions, thus giving a framework to prioritise the interventions
 \item The proportion of NI Act cases is significant (13.2\%) and highlights why it is necessary to support policy decisions with robust empirical studies.
 \item It also demonstrates the need for courts to institute mechanisms to record better case characteristics and case flow data. 
 \item Courts, in turn, can use these analyses to improve procedures and administration and ensure better justice delivery.
 \end{itemize}
\end{frame}

\begin{frame}
 \centering \huge{Thank you}
\end{frame}

\end{document}