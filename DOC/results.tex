Determinants of case duration in NI Act cases, their incidence and impact
\begin{enumerate}
\item Accused absconding
\item Duration and outcome of mediation
\item Summary trial being converted to summons trial
\item Incidence of summary trial being converted to summons trial
  before assessing whether accused has a plausible defence
\item Wrong jurisdiction
\item Multiple cheques bounced in the same transaction
\end{enumerate}


\begin{longtable}{@{}lllllll@{}}
\caption{Impact of case characteristics on case duration (in days)}
\label{tab:duration_reg}\\
\toprule
variable & coeff & std err & t & P>|t| & [0.25 & 0.95] \\\midrule
\endhead
Intercept & 407.2465 & 10.065 & 40.462 & 0.000 & 387.519 & 426.974 \\
C(stateName)[T.Chandigarh] & -266.5441 & 16.731 & -15.931 & 0.000 & -299.337 & -233.751 \\
C(stateName)[T.Delhi] & 98.8085 & 10.513 & 9.399 & 0.000 & 78.204 & 119.413 \\
C(stateName)[T.Goa] & -131.1721 & 24.274 & -5.404 & 0.000 & -178.750 & -83.594 \\
C(stateName)[T.Gujarat] & -161.1733 & 10.013 & -16.097 & 0.000 & -180.799 & -141.548 \\
C(stateName)[T.Haryana] & -235.2128 & 10.458 & -22.492 & 0.000 & -255.710 & -214.715 \\
C(stateName)[T.Himachal Pradesh] & 3.8897 & 16.965 & 0.229 & 0.819 & -29.362 & 37.141 \\
C(stateName)[T.Karnataka] & -121.5457 & 9.110 & -13.342 & 0.000 & -139.402 & -103.690 \\
C(stateName)[T.Maharashtra] & -10.1451 & 10.067 & -1.008 & 0.314 & -29.876 & 9.586 \\
C(stateName)[T.Punjab] & -253.8969 & 10.164 & -24.980 & 0.000 & -273.819 & -233.975 \\
C(year)[T.2015] & 25.6553 & 6.816 & 3.764 & 0.000 & 12.295 & 39.015 \\
C(year)[T.2016] & 9.7630 & 6.561 & 1.488 & 0.137 & -3.097 & 22.623 \\
C(year)[T.2017] & -100.5308 & 6.510 & -15.442 & 0.000 & -113.291 & -87.770 \\
C(year)[T.2018] & -192.9984 & 6.816 & -28.315 & 0.000 & -206.358 & -179.638 \\
charNonAppearance & 213.3056 & 4.855 & 43.932 & 0.000 & 203.789 & 222.822 \\
charSummons & 111.5809 & 5.141 & 21.704 & 0.000 & 101.504 & 121.658 \\
charMediation & 108.0148 & 4.957 & 21.792 & 0.000 & 98.300 & 117.730 \\
charJurisdiction & 286.8096 & 4.922 & 58.270 & 0.000 & 277.162 & 296.457 \\
charMultipleCheques & 171.0771 & 9.938 & 17.215 & 0.000 & 151.599 & 190.555 \\
contested & -46.5545 & 5.244 & -8.877 & 0.000 & -56.834 & -36.275\\
\bottomrule
No. Observations & 37175 & & & & &\\
R-squared & 0.258 & & & & & \\
Adj. R-squared& 0.258& & & & & \\
Df Residuals& 37155 & & & & &\\
F-statistic & 680.7 & & & & & \\
Log-Likelihood & -2.7359e+05 & & & & & \\
\bottomrule
\end{longtable}

\begin{itemize}
\item Non appearance and jurisdiction issues can add more than 200
  days to the duration. 
\item Getting referred to mediation, conversion to
  summons trials and multiplicity can add more than 100 days to the
  duration. 
\item Cases that are contested, however, seem to take approx 40
  days less than the rest. This finding is unexpected.
\end{itemize}


\begin{longtable}{@{}lllllll@{}}
\caption{Impact of case characteristics on number of hearings to dispose}
\label{tab:hearings_reg}\\
\toprule
variable & coeff & std err & t & P>|t| & [0.25 & 0.95] \\\midrule
\endhead
%
Intercept & 9.8428 & 0.260 & 37.863 & 0.000 & 9.333 & 10.352 \\
C(stateName)[T.Chandigarh] & -12.2753 & 0.432 & -28.406 & 0.000 & -13.122 & -11.428 \\
C(stateName)[T.Delhi] & -7.1574 & 0.272 & -26.360 & 0.000 & -7.690 & -6.625 \\
C(stateName)[T.Goa] & -5.6849 & 0.627 & -9.067 & 0.000 & -6.914 & -4.456 \\
C(stateName)[T.Gujarat] & -5.0186 & 0.259 & -19.405 & 0.000 & -5.526 & -4.512 \\
C(stateName)[T.Haryana] & -11.5331 & 0.270 & -42.698 & 0.000 & -12.063 & -11.004 \\
C(stateName)[T.Himachal Pradesh] & -7.0440 & 0.438 & -16.076 & 0.000 & -7.903 & -6.185 \\
C(stateName)[T.Karnataka] & -5.9616 & 0.235 & -25.336 & 0.000 & -6.423 & -5.500 \\
C(stateName)[T.Maharashtra] & -2.6824 & 0.260 & -10.317 & 0.000 & -3.192 & -2.173 \\
C(stateName)[T.Punjab] & -8.6937 & 0.263 & -33.116 & 0.000 & -9.208 & -8.179 \\
C(year)[T.2015] & 1.0210 & 0.176 & 5.799 & 0.000 & 0.676 & 1.366 \\
C(year)[T.2016] & -0.1450 & 0.169 & -0.856 & 0.392 & -0.477 & 0.187 \\
C(year)[T.2017] & -2.3964 & 0.168 & -14.251 & 0.000 & -2.726 & -2.067 \\
C(year)[T.2018] & -4.0021 & 0.176 & -22.732 & 0.000 & -4.347 & -3.657 \\
charNonAppearance & 7.0313 & 0.125 & 56.068 & 0.000 & 6.785 & 7.277 \\
charSummons & 7.1786 & 0.133 & 54.060 & 0.000 & 6.918 & 7.439 \\
charMediation & 3.2714 & 0.128 & 25.552 & 0.000 & 3.020 & 3.522 \\
charJurisdiction & 5.6611 & 0.127 & 44.530 & 0.000 & 5.412 & 5.910 \\
charMultipleCheques & 9.9920 & 0.257 & 38.928 & 0.000 & 9.489 & 10.495 \\
contested & 2.8939 & 0.135 & 21.364 & 0.000 & 2.628 & 3.159
\bottomrule
No. Observations & 37175 & & & & &\\
R-squared & 0.373 & & & & & \\
Adj. R-squared& 0.373& & & & & \\
Df Residuals& 37155 & & & & &\\
F-statistic & 1163 & & & & & \\
Log-Likelihood & -1.3767e+05 & & & & & \\
\bottomrule
\end{longtable}

\begin{itemize}
\item Cases with non appearance and conversion to summons trials need
  approx 7 more hearings to dispose than the rest 
\item Cases which get
  referred to mediation take 3 more hearings than the rest 
\item Cases with
  jurisdictional issues take 5 hearings more than the rest 
\item Cases with
  multiplicity of offences can take more than 9 hearings extra 
\item Cases
  that are contested, however, seem to take 2-3 hearings more to
  dispose than uncontested cases
\end{itemize}

%%% Local Variables:
%%% mode: latex
%%% TeX-master: "paper_chequeDishonour"
%%% End:
