\subsection{Determinants of total duration} \label{sec:determ-total-durat}

Table \ref{tab:duration_reg} in Appendix \ref{sec:impact-case-char} shows the regression model to measure the effect of case characteristics on the case duration (in days). The state in which the case is filed has a large and statistically significant effect on the case duration.

Controlling for the state-level effects, all but one of the selected case characteristics viz. -- non-appearance of accused, case being converted to a summons trial, reference to mediation, jurisdictional issues, and multiplicity of proceedings -- significantly increase the case duration. As Table \ref{tab:duration_reg} shows, \emph{ceteris paribus}:

\begin{enumerate}
\item non-appearance of the accused typically adds 213 days to the total case duration;
\item conversion to a summons trial typically adds 111 days;
\item the case being referred to mediation typically adds 108 days;
\item jurisdictional issues typically add 287 days; and
\item multiplicity of proceedings typically adds 171 days.
\end{enumerate}

Notably, cases referred to mediation take longer to dispose than cases not referred to mediation. This runs counter to our hypothesis that cases referred to mediation take less time to dispose. This also means that the recommendation by the Amicus Curiae, calling for more cases to be referred to mediation, will not reduce the delays in cheque-dishonour cases. They will increase the delays by 108 days (all else being equal).

Further, \emph{ceteris paribus}, uncontested cases typically take 46 days more to dispose than contested cases. This could indicate that there might be some scope for improving the scheduling mechanism for uncontested cases.

\subsection{Determinants of the number of hearings to dispose} \label{sec:determ-numb-hear}

Table \ref{tab:hearings_reg} in Appendix \ref{sec:impact-case-char-1} shows the regression model to measure the effect of case characteristics on the number of hearings required to dispose the case. The state in which the case is filed has a large and statistically significant effect on the number of hearings. Controlling for the state-level effects, all of the selected case characteristics viz. --- non-appearance of accused, case being converted to a summons trial, reference to mediation, jurisdictional issues, and multiplicity of proceedings --- significantly increase the number of hearings required to dispose the case. As Table \ref{tab:hearings_reg} shows, \emph{ceteris paribus}:

\begin{enumerate}
\item non-appearance of the accused typically adds 7 hearings to the total hearings required to dispose the case;
\item conversion to a summons trial typically adds 7.2;
\item the case being referred to mediation typically adds 3.3 hearings;
\item jurisdictional issues typically add 5.7 hearings; and
\item multiplicity of proceedings typically adds 10 hearings.
\end{enumerate}

Notably, cases referred to mediation take more hearings to dispose than cases not referred to mediation. This runs counter to our hypothesis that cases referred to mediation take fewer hearings to dispose. Read with the finding on duration, it means cases referred to mediation not only take longer to dispose, but they also do not reduce the amount of time the court has to allocate to them. This also means that the recommendation by the Amicus Curiae, calling for more cases to be referred to mediation, will not increase the time available for hearing other matters. This points to some fundamental issues in the court-ordered mediation process and the functioning of mediation and conciliation forums such as Lok Adalats.

Another notable finding is that contested cases take more hearings to dispose. Read with the earlier finding of contested cases, the courts efficiently dispose cases when both parties participate in the proceedings. It also indicates that when the parties contest, the court schedules successive hearings in a shorter time than when the matter is uncontested.

\subsection{Further examining cases referred to mediation} \label{sec:furth-exam-cases}

Table \ref{tab:mediation} summarises the outcomes of cases referred to mediation. We can count all instances where a case is settled, withdrawn or compounded as successful mediation. These constitute 78.6\% of cases referred to mediation. In other words, the mediation process concludes in a successful mediation in a vast majority of the cases. The case gets sent back to the court for adjudication only in 21.4\% of cases. Read with the result on the duration of cases referred to mediation being longer than other cases; this means that the delays result from issues in the mediation process itself. Precisely identifying issues with the mediation process requires further investigation.

{\footnotesize \begin{longtable}{@{}clrrr@{}}
\caption{Outcomes of cases referred to mediation}
\label{tab:mediation}\\
\toprule
 \textbf{Disposal type} & \multicolumn{1}{c}{\textbf{Disposal sub-type}} & \multicolumn{1}{c}{\textbf{Total cases}} & \multicolumn{1}{c}{\textbf{Percentage}} & \multicolumn{1}{p{3cm}}{\textbf{Median duration (in days)}} \\
 \midrule
\endhead
\multirow{3}{*}{Dismissed} & other & 478 & 5.46 & 710 \\
 & settlement & 8 & 0.09 & 512.5 \\
 & withdrawn & 2942 & 33.58 & 491 \\
 \midrule
\multirow{4}{*}{Disposed} & compounded & 459 & 5.24 & 638 \\
 & other & 1395 & 15.92 & 723 \\
 & settlement & 2945 & 33.62 & 527 \\
 & withdrawn & 213 & 2.43 & 462 \\
 \midrule
 Other & other & 320 & 3.65 & 752.5 \\
 \midrule
 \multicolumn{2}{c}{\textbf{Total}} & \textbf{8760} & \textbf{100} & \multicolumn{1}{l}{\textbf{-}} \\
 \bottomrule
 \multicolumn{5}{p{11cm}}{{\footnotesize \emph{Note: The total is less than the total cases referred to mediation owing to the limitations in the data on disposal type and filing/disposal dates.}}}
\end{longtable}}

\subsection{Potential impact on the case-loads} \label{sec:impact-case-loads}

Table \ref{tab:summary_results} shows a summary of the results of the analysis. The delays resulting from the respective characteristics affect a significant proportion of \gls{ni} cases. Further, since \gls{ni} cases constitute 13.25\% of the total case load, these delays are bound to contribute to the overall delays in courts. They have a significant effect on the overall pendency. In particular, if the problem of the accused not appearing in the court can be solved, all else being equal, it will speed up the disposal of more than 9\% of the total case load by 213 days, and courts will have to allocate 7 fewer hearings to dispose each of them.

{\footnotesize \begin{longtable}{@{}p{2.5cm}rrrrr}
 \caption{Summary of results}\label{tab:summary_results}\\
 \toprule
 \textbf{Characteristic} & \multicolumn{1}{p{2cm}}{\textbf{Number of cases}} &
 \multicolumn{1}{p{2cm}}{\textbf{As percentage of NI Act cases}}
 & \multicolumn{1}{p{2cm}}{\textbf{As percentage of total cases}}
 & \multicolumn{1}{p{2cm}}{\textbf{Effect on days to dispose}} &
 \multicolumn{1}{p{2cm}}{\textbf{Effect on hearings to dispose}}
 \\
 \midrule
 Non appearance of the accused & 33625 & 69.80\% & 9.24\% & +213 & +7.03 \\ \midrule
 Conversion to summons trial & 12876 & 26.70\% & 3.54\% & +111 & +7.18 \\ \midrule
 Mediation & 10711 & 22.20\% & 2.94\% & +108 & +3.27 \\ \midrule
 Jurisdictional issues & 14098 & 29.20\% & 3.88\% & +287 & +5.66 \\ \midrule
 Multiplicity of proceedings & 2010 & 4.20\% & 0.55\% & +171 & +9.99 \\ \midrule
 Contested & 8283 & 17.20\% & 2.28\% & --46 & +2.89 \\ \midrule
 Total & 48191 & 100.00\% & 13.25\% & N/A & N/A \\
 \bottomrule
 \multicolumn{6}{l}{{\footnotesize \emph{Note: `+' sign
  indicates an increase, while `-' sign indicates decrease.}}}\\
\end{longtable}
}

According to the NJDG, as of January 2022, there were 3,57,72,846 total original pending cases in the subordinate courts. Assuming the proportions and ratios hold good for the country as a whole, 13.25\% of these would be \gls{ni} cases, which amounts to 47,39,902 cases. To put the numbers in the table into context, ensuring the accused's presence can lead to a savings of 2,32,58,415 hearings across the courts in the country.\footnote{This assumes the proportion of the distribution of case characteristics holds good for the country as a whole.} Similarly, if all of the \gls{ni} cases are disposed summarily, courts across the country can avoid a total of 90,86,676 hearings.

In the same vein, referring \gls{ni} cases to mediation could be adding 34,40,884 hearings to the courts across the country. All those hearings could be avoided if these cases were instead tried and disposed by the court itself. Conversely, if we assume that all the cases were to be referred to mediation, it would add a total of 1,54,99,479 total hearings. Further, these cases would also require 108 more days each to dispose, on average. This is significant and highlights why it is necessary to support policy decisions with robust empirical studies.

%%% Local Variables:
%%% mode: latex
%%% TeX-master: "paper_chequeDishonour"
%%% End: